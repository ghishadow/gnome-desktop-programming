\chapter{Strings}

TODO: 

\section{Building Strings}

GLib comes with a structure to help build strings called \verb|GString|.
The purpose of \verb|GString| is to allow you to continually append characters or chunks of strings until you have what you want.
After which, you free the structure but retain the resulting string.
This saves you from continually having to allocate and free parts of a string.

\begin{code}{}
gchar *
build_string (void)
{
  GString *str
  gint i;

  str = g_string_new(NULL);
  for (i = 99; i >= 0; i--) {
    g_string_append_printf(str, "%d bottles of beer\n", i);
  }

  return g_string_free(str, FALSE);
}
\end{code}

Notice line 12 and the call to \verb|g_string_free()|.
The second parameter here denotes that the built string should not be freed with the \verb|GString| structure.
Instead, \verb|g_string_free()| will return the string.
This is convenient for the example shown above as we can simply return from our function providing that string.

As you are building your string, \verb|GString| is managing the memory required to contain it.
To be efficient, every time it allocates more memory it does so by growing in powers of two.
This means that if you had 32 bytes in your string, and added another byte, it would allocate 64 bytes.
This helps when small additions are made because it reduces the total number of memory allocations.

If you do not need to format the next chunk of the string then use \verb|g_string_append()|.

\section{Unicode}

\section{Character Set Conversion}

