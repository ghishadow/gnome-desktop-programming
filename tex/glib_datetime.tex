\chapter{Dates and Times}

At some point you will need to work with dates and times.
Every operating system has its own quirks in dealing with times.
For this reason, GLib has a couple of abstraction layers dependong on what you need to do.

If what you need is to stick close to the posix \verb|struct tv|, there is \verb|GTimeVal|.
It is also useful if you need to work on a large array of dates and times.
\verb|GDateTime| was added recently to provide a higher-level API this is more convenient in both C and higher level languages.
As one would guess, \verb|GTimeZone| provides information about a timezone such as its GMT offset and if it is daylight savings.

\section{GTimeVal}

\verb|GTimeVal| is a structure that contains two fields.
The first is \verb|tv_sec| and it contains the number of seconds since January 1, 1970.
The second is \verb|tv_usec| and it contains the number of microseconds past the current second.

\marginpar{
    Notice that tv\_sec and tv\_usec are glong.
    Performing arithmetic may overflow on 32-bit architectures sooner than you think!
}
\begin{code}{}
typedef struct
{
    glong tv_sec;
    glong tv_usec;
} GTimeVal;
\end{code}

To get the current time using \verb|GTimeVal|, we use the \verb|g_get_current_time()| function.
It asks the system clock for the current local time and stores it in the structure provided.
You are responsible for that memory.
In this case, it is simply allocated on the stack.
However, it could just as easily be a field in a structure of your own.

\begin{code}{}
GTimeVal tv;
g_get_current_time(&tv);
\end{code}

It is common to covert dates and times to a string for communication with external systems.
ISO-8601 is a very commonly supported format for doing this.
It contains the date, time, and timezone offset and looks something like \verb|2012-11-12T23:59:02-0800|.
GLib provides functions to translate to and from this format.

\begin{code}{}
GTimeVal tv;
gchar *tvstr;

g_get_current(&tv);
tvstr = g_time_val_to_iso8601(tvstr);
g_time_val_from_iso8601(&tv, tvstr);
g_free(tvstr);
\end{code}


\section{GDateTime}



\section{GTimeZone}


\section{Miscellaneous}

TODO

\begin{itemize}
\item \verb|g_usleep()|
\item \verb|g_get_monotonic_time()|
\item \verb|g_get_real_time()|
\end{itemize}
